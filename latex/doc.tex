%%%%%%%%%%%%%%%%%%%%%%%%%%%%%%%%%%%%%%%%%
% Arsclassica Article
% LaTeX Template
% Version 1.1 (1/8/17)
%
% This template has been downloaded from:
% http://www.LaTeXTemplates.com
%
% Original author:
% Lorenzo Pantieri (http://www.lorenzopantieri.net) with extensive modifications by:
% Vel (vel@latextemplates.com)
%
% License:
% CC BY-NC-SA 3.0 (http://creativecommons.org/licenses/by-nc-sa/3.0/)
%
%%%%%%%%%%%%%%%%%%%%%%%%%%%%%%%%%%%%%%%%%

%----------------------------------------------------------------------------------------
%	PACKAGES AND OTHER DOCUMENT CONFIGURATIONS
%----------------------------------------------------------------------------------------

\documentclass[
10pt, % Main document font size
a4paper, % Paper type, use 'letterpaper' for US Letter paper
oneside, % One page layout (no page indentation)
%twoside, % Two page layout (page indentation for binding and different headers)
headinclude,footinclude, % Extra spacing for the header and footer
BCOR5mm, % Binding correction
]{scrartcl}

%%%%%%%%%%%%%%%%%%%%%%%%%%%%%%%%%%%%%%%%%
% Arsclassica Article
% Structure Specification File
%
% This file has been downloaded from:
% http://www.LaTeXTemplates.com
%
% Original author:
% Lorenzo Pantieri (http://www.lorenzopantieri.net) with extensive modifications by:
% Vel (vel@latextemplates.com)
%
% License:
% CC BY-NC-SA 3.0 (http://creativecommons.org/licenses/by-nc-sa/3.0/)
%
%%%%%%%%%%%%%%%%%%%%%%%%%%%%%%%%%%%%%%%%%

%----------------------------------------------------------------------------------------
%	REQUIRED PACKAGES
%----------------------------------------------------------------------------------------

\usepackage[
nochapters, % Turn off chapters since this is an article        
beramono, % Use the Bera Mono font for monospaced text (\texttt)
eulermath,% Use the Euler font for mathematics
pdfspacing, % Makes use of pdftex’ letter spacing capabilities via the microtype package
dottedtoc % Dotted lines leading to the page numbers in the table of contents
]{classicthesis} % The layout is based on the Classic Thesis style

\usepackage{arsclassica} % Modifies the Classic Thesis package

\usepackage[T1]{fontenc} % Use 8-bit encoding that has 256 glyphs

\usepackage[utf8]{inputenc} % Required for including letters with accents

\usepackage{graphicx} % Required for including images
\graphicspath{{Figures/}} % Set the default folder for images

\usepackage{enumitem} % Required for manipulating the whitespace between and within lists

\usepackage{lipsum} % Used for inserting dummy 'Lorem ipsum' text into the template

\usepackage{subfig} % Required for creating figures with multiple parts (subfigures)

\usepackage{amsmath,amssymb,amsthm} % For including math equations, theorems, symbols, etc

\usepackage{varioref} % More descriptive referencing

%----------------------------------------------------------------------------------------
%	THEOREM STYLES
%---------------------------------------------------------------------------------------

\theoremstyle{definition} % Define theorem styles here based on the definition style (used for definitions and examples)
\newtheorem{definition}{Definition}

\theoremstyle{plain} % Define theorem styles here based on the plain style (used for theorems, lemmas, propositions)
\newtheorem{theorem}{Theorem}

\theoremstyle{remark} % Define theorem styles here based on the remark style (used for remarks and notes)

%----------------------------------------------------------------------------------------
%	HYPERLINKS
%---------------------------------------------------------------------------------------

\hypersetup{
%draft, % Uncomment to remove all links (useful for printing in black and white)
colorlinks=true, breaklinks=true, bookmarks=true,bookmarksnumbered,
urlcolor=webbrown, linkcolor=RoyalBlue, citecolor=webgreen, % Link colors
pdftitle={}, % PDF title
pdfauthor={\textcopyright}, % PDF Author
pdfsubject={}, % PDF Subject
pdfkeywords={}, % PDF Keywords
pdfcreator={pdfLaTeX}, % PDF Creator
pdfproducer={LaTeX with hyperref and ClassicThesis} % PDF producer
} % Include the structure.tex file which specified the document structure and layout

\hyphenation{Fortran hy-phen-ation} % Specify custom hyphenation points in words with dashes where you would like hyphenation to occur, or alternatively, don't put any dashes in a word to stop hyphenation altogether

\newcommand{\bftheta}{\boldsymbol{\theta}}
\newcommand{\bfmu}{\boldsymbol{\mu}}
\newcommand{\bfalpha}{\boldsymbol{\alpha}}
\newcommand{\bfX}{\boldsymbol{X}}
\newcommand{\bfx}{\boldsymbol{x}}
\newcommand{\bfw}{\boldsymbol{w}}
\newcommand{\bfY}{\boldsymbol{Y}}
\newcommand{\bfy}{\boldsymbol{y}}
\newcommand{\bfz}{\boldsymbol{z}}
\newcommand{\bfbeta}{\boldsymbol{\beta}}
\newcommand{\bfpi}{\boldsymbol{\pi}}
\newcommand{\bfe}{\boldsymbol{e}}
\newcommand{\bff}{\boldsymbol{f}}
\DeclareMathOperator{\tr}{tr}

\newtheorem{remark}{Remark}

%----------------------------------------------------------------------------------------
%	TITLE AND AUTHOR(S)
%----------------------------------------------------------------------------------------

\title{\normalfont\spacedallcaps{Learning with Random Inputs}} % The article title

%\subtitle{Subtitle} % Uncomment to display a subtitle

\author{\spacedlowsmallcaps{Robert Wolstenholme}} % The article author(s) - author affiliations need to be specified in the AUTHOR AFFILIATIONS block

\date{} % An optional date to appear under the author(s)

%----------------------------------------------------------------------------------------

\begin{document}

%----------------------------------------------------------------------------------------
%	HEADERS
%----------------------------------------------------------------------------------------

\renewcommand{\sectionmark}[1]{\markright{\spacedlowsmallcaps{#1}}} % The header for all pages (oneside) or for even pages (twoside)
%\renewcommand{\subsectionmark}[1]{\markright{\thesubsection~#1}} % Uncomment when using the twoside option - this modifies the header on odd pages
\lehead{\mbox{\llap{\small\thepage\kern1em\color{halfgray} \vline}\color{halfgray}\hspace{0.5em}\rightmark\hfil}} % The header style

\pagestyle{scrheadings} % Enable the headers specified in this block

%----------------------------------------------------------------------------------------
%	TABLE OF CONTENTS & LISTS OF FIGURES AND TABLES
%----------------------------------------------------------------------------------------

\maketitle % Print the title/author/date block

\setcounter{tocdepth}{2} % Set the depth of the table of contents to show sections and subsections only

\tableofcontents % Print the table of contents

\listoffigures % Print the list of figures

\listoftables % Print the list of tables

%----------------------------------------------------------------------------------------
%	ABSTRACT
%----------------------------------------------------------------------------------------

\section*{Abstract} % This section will not appear in the table of contents due to the star (\section*)

Given some observations of dependent and independent and independent variables, there are many ways to learn some relationship between them. This essentially falls under the category of supervised learning. However, if we do not directly observe the independent variables but instead simply the distribution they have come from we have to modify some techniques so that they become tractable. This can arise when the true values of the independent variables are essentially hidden and we only have an estimate of their distribution.

We will examine the case for a softmax regression with a multivariate normal distribution for the independent varaibles. Two different approximations are used to the objective function to make the optimisation tractable. Then we look to performing the optimisation in the Scikit-Learn and Tensorflow libraries.

%----------------------------------------------------------------------------------------
%	AUTHOR AFFILIATIONS
%----------------------------------------------------------------------------------------

\let\thefootnote\relax\footnotetext{* \textit{Department of Biology, University of Examples, London, United Kingdom}}

\let\thefootnote\relax\footnotetext{\textsuperscript{1} \textit{Department of Chemistry, University of Examples, London, United Kingdom}}

%----------------------------------------------------------------------------------------

\newpage % Start the article content on the second page, remove this if you have a longer abstract that goes onto the second page

%----------------------------------------------------------------------------------------
%	INTRODUCTION
%----------------------------------------------------------------------------------------

\section{Introduction}

Consider the known multivariate normal parameters at time $t$

$$ \bfmu_t \in \mathbb{R}^n \mbox{ and } \Sigma_t \in \mathbb{R}^{n \times n} $$

such that we have a random variable

$$ \bfX_t \sim MVN(\bfmu_t, \Sigma_t). $$

\begin{remark}
The actual realisations $\bfx_t$ from $\bfX_t$ will never be observed and essentially represent independent variables.
\end{remark}

Also consider an observation vector at time $t$,

$$ \bfy_t \in \{0,1\}^k $$

such that $\sum_{i=0}^k y_{i,t} = 1$ i.e. $\bfy_t$ contains a unique entry equal to 1 and the rest are 0. It therefore represents our dependent variable and is a realisation from a multinomial distribution conditional on $\bfX_t$,

$$ \bfY_t | \bfX_t = \bfx_t \sim Multinomial(g_W(\bfx_t)), $$

where the $i$th component of $g_W(\bfx_t)$ is

$$ [g_W(\bfx_t)]_i = \frac{\exp(\bfw_i^T \bfx_t)}{\sum_{j=1}^k \exp(\bfw_j^T \bfx_t)}.$$

Hence $g_W(\cdot)$ is a softmax transformation with weight vectors $\bfw_i \in \mathbb{R}^n$ which we write as a matrix

$$ W = [\bfw_1, \dots, \bfw_k] \in \mathbb{R}^{n \times k}. $$

Our goal is that given some sequence of observations $\{\bfy_t\}$ and observed distribution parameters $(\{\bfmu_t\}, \{\Sigma_t\})$, to find optimal weight matrix $W$.

\begin{remark}
While we think of all components of $\bfX_t$ being stochastic, non-random components are of course dealt with by $0$s in the covariance matrix.
\end{remark} 

 
%----------------------------------------------------------------------------------------
%	Optimisation
%----------------------------------------------------------------------------------------

\section{Optimisation}
In order to write the optimisation, we must first write the likelihood function, given some values for $W$, of observing the $\bfy_t$ value.

The conditional probability is (from a multinomial distribution),

$$ \Pr(\bfY_t = \bfy_t | \bfX_t = \bfx_t; W) = g_W(\bfx_t)^T \bfy_t $$

and so the likelihood is

$$ \Pr(\bfY_t = \bfy_t; W) = \int_{\bfx_t} \left [g_W(\bfx_t)^T \bfy_t \right ] f_{\bfX_t}(\bfx_t) d\bfx_t $$ 

where $f_{\bfX_t}(\bfx_t)$ is a multivariate normal pdf with parameters $\bfmu_t$ and $\Sigma_t$ and the `$;W$' represents the fact the probability depends on deterministic matrix $W$.

Expanding this over all values of $t$ ($1, \dots, T$), we have likelihood

$$ L(W) = \prod_{t=1}^T \Pr(\bfY_t = \bfy_t; W) $$
$$ = \int_{\bfx_1} \dots \int_{\bfx_T} \prod_{t=1}^T \left [g_W(\bfx_t)^T \bfy_t \right ] f_{\bfX_1,\dots,\bfX_T}(\bfx_1, \dots, \bfx_T) d\bfx_1 \dots d\bfx_T. $$

Under the assumption of independence between the $\{\bfX_t\}$ (which in reality is very often not going to be true) this can be written

$$ L(W) = \prod_{t=1}^T \int_{\bfx_t} \left [g_W(\bfx_t)^T \bfy_t \right ] f_{\bfX_t}(\bfx_t) d\bfx_t$$

and the log likelihood can be written

$$ \log(L(W)) = LL(W) = \sum_{t=1}^T \log \left [ \int_{\bfx_t} \left [g_W(\bfx_t)^T \bfy_t \right ] f_{\bfX_t}(\bfx_t) d\bfx_t \right ].$$

The optimisation to solve can then be written as

$$ \hat{W} = \arg \max_W LL(W) + \lambda r(W)$$

for some $\lambda \in \mathbb{R}$ and regularisation function $r: \mathbb{R}^{n \times k} \rightarrow \mathbb{R}$.

\begin{remark}
\begin{enumerate}
  \item Despite the fact the independence between the $\{\bfX_t\}$ is often not true, there is very little we can do. If we don't know the joint distribution of course there is nothing we can do, but even if it is known, the already intractable computation becomes even harder to approximate.
  \item The integrals within the likelihood are expectations over $\bfX_t$ i.e.
  $$ \mathbb{E}_{\bfX_t} \left [ g_W(\bfx_t)^T \bfy_t \right ] = \int_{\bfx_t} \left [g_W(\bfx_t)^T \bfy_t \right ] f_{\bfX_t}(\bfx_t) d\bfx_t.$$
  \item In general the integral is intractable and we have to use numerical approximations to evaluate and get a gradient for $LL(W)$.
  \item For the regularisation, we will use the L2 norm i.e. 
  $$r(W) = || W ||_2 = \sum_{i,j} W_{ij}^2.$$
\end{enumerate}
\end{remark}

\subsection{Approximation 1}
First write the log likelihood as 
$$ LL(W) = \sum_{t=1}^T \log \mathbb{E}_{\bfX_t} \left [g_W(\bfx_t)^T \bfy_t \right ]. $$

Now, $g_W(\bfx_t)^T \bfy_t$ is not convex or concave. This can be seen by considering the sigmoid function $y = \frac{e^x}{1+e^x}$ which has second derivative $\frac{\partial^2 y}{\partial x^2} = (1-y)(1-2y)y$. This is negative for $x > 0$ and positive for $x < 0$.

We cannot therefore use Jensen's inequality to get a lower/upper bound but we still attempt the approximation by swapping the expectation into the function

$$ LL^{(1)}(W) = \sum_{t=1}^T \log \left [g_W(\mathbb{E}_{\bfX_t}[\bfx_t])^T \bfy_t \right ] = \sum_{t=1}^T \log \left [g_W(\bfmu_t)^T \bfy_t \right ].$$

The optimsation
$$ \hat{W}^{(1)} = \arg \max_W LL^{(1)}(W) + \lambda r(W) $$

is simply a standard constrained softmax regression and we solve it using:
\begin{enumerate}
  \item Scikit-Learn: sklearn.linear\_model.LogisticRegression class with multi\_class = `multinomial'.
  \item Tensorflow: Coded constrained softmax regression.
\end{enumerate}

\begin{remark}
The above approximation completely discard the information provided by the $\{\Sigma_t\}$ covariance matrices.
\end{remark}

\subsection{Approximation 2}
For our second approximation, we approximate the integrals as finite sums by samping from the $f_{\bfX_t}$ distribution i.e.

$$ \int_{\bfx_t} \left [g_W(\bfx_t)^T \bfy_t \right ] f_{\bfX_t}(\bfx_t) d\bfx_t \approx \frac{1}{| \mathbb{X}_t |}\sum_{\bfx \in \mathbb{X}_t} g_W(\bfx)^T \bfy_t $$

where $\mathbb{X}_t$ is a set of samples from the distribution with pdf $f_{\bfX_t}$.

Therefore we write
$$ LL^{(2)}(W) = \sum_{t=1}^T \log \left [ \frac{1}{| \mathbb{X}_t |}\sum_{\bfx \in \mathbb{X}_t} g_W(\bfx)^T \bfy_t \right ].$$

Note that this has gradient
$$ \frac{\partial LL^{(2)}(W)}{\partial W} = \sum_{t=1}^T  \frac{\sum_{\bfx \in \mathbb{X}_t} \frac{\partial g_W(\bfx)^T \bfy_t}{\partial W}} {\sum_{\bfx \in \mathbb{X}_t} g_W(\bfx)^T \bfy_t} .$$

\begin{remark}
\begin{enumerate}
  \item The size of the sets $\mathbb{X}_t$ no longer matters once we consider
        the derivative of $LL^{(2)}(W)$ (which is what will be used in gradient
        ascent/descent). Hence all that matters is we have enough samples for a
        `good' approximation to each integral. Even then the definition of good
        really is whether the optimisation converges to the correct values and
        if we have enough observations, it may be enough to simply have $|
        \mathbb{X}_t | = 1$!
  \item In some border cases however we must make very sure we have enough
        observations in our set for a good prediction. Consider $n=1$, $k=2$
        and $g_W(x) = [I(Wx > 2), I(Wx \leq 2)]^T$. Then for $x \leq W/2$ we
        predict class $1$ and for $x > W/2$ we predict class $0$. Hence if the
        true $x$ value caused a class $0$ prediction but a single sampled value
        caused a class $1$ prediction, then the true parameter value $W$ will
        cause an estimated log likelihood of $- \infty$! Even worse, if this
        happened multiple times, it may be impossible to have any non
        degenerate value for the estimated value of $W$. Of course the model we
        use for $g_W$ is a multinomial model so the above definition is
        impossible. However we can get similar situation arising if the
        distribution of x has a very large variance. For example, the observed
        class may be class $0$, but the single sampled value may have a very
        large magnitude making the prediction of class $1$ occur with very high
        probability. This can be enough to significantly affect the overall
        log likelihood and hence cause a bad $W$ estimation.
  \item If we have $| \mathbb{X}_t | = 1 $ for all $t$, then $LL^{(2)}(W)$ is
        in the form of a standard softmax regression, like $LL^{(1)}(W)$, with
        the only difference being that $LL^{(1)}(W)$ uses the mean value of the
        observations $\bfmu_t$ and $LL^{(2)}(W)$ uses a sample from
        $MVN(\bfmu_t, \Sigma_t)$.
\end{enumerate}
\end{remark}

Hence, we solve $LL^{(2)}(W)$ using:
\begin{enumerate}
  \item Scikit-Learn: sklearn.linear\_model.LogisticRegression class with multi\_class = `multinomial' only when $| \mathbb{X}_t | = 1$ for all $t$.
  \item Tensorflow: For varying sizes of the $\mathbb{X}_t $.
\end{enumerate}


%----------------------------------------------------------------------------------------
%	BIBLIOGRAPHY
%----------------------------------------------------------------------------------------

\renewcommand{\refname}{\spacedlowsmallcaps{References}} % For modifying the bibliography heading

\bibliographystyle{unsrt}

\bibliography{sample.bib} % The file containing the bibliography

%----------------------------------------------------------------------------------------

\end{document}
